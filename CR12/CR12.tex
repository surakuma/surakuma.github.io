\documentclass[9pt]{beamer}
\usepackage[utf8]{inputenc}
\usepackage[french]{babel}
\usepackage{graphicx}
\usepackage{beamerthemesplit}
\usetheme{Ilmenau}
%\usetheme{Warsaw}%\usetheme{Frankfurt}

\usepackage{subfigure}
\usepackage{booktabs}
\usepackage[utf8]{inputenc}
\usepackage{helvet}
\usepackage{multimedia}
\usepackage{media9}
\usepackage{stmaryrd}
\usepackage{xpatch}
\usepackage[absolute,overlay]{textpos}
\usepackage{siunitx}
\usepackage{pgfplots}
\usepackage{pgfplotstable}
\usepackage{listings}
\usepackage{algorithm,algorithmic}
%\usepackage{algpseudocode}
\usepackage{amsmath}
\usepackage{amssymb}
\usepackage{amsthm}
\usepackage{hyperref}
\usepackage{tikz}
\usepackage{fancybox}
\usepackage{multirow}
\usetikzlibrary{calc}
\usetikzlibrary{arrows}
\usetikzlibrary{shapes.geometric}
\usetikzlibrary{decorations.pathreplacing}
\usetikzlibrary{patterns,decorations.pathreplacing}
\usetikzlibrary{matrix}
\usetikzlibrary{fit}
\usetikzlibrary{trees}
\usetikzlibrary{backgrounds}
\usetikzlibrary{patterns}
\usetikzlibrary{positioning}
\usetikzlibrary{decorations.pathreplacing}
\usepackage{blindtext}
\usepackage{pgfplots}
\pgfplotsset{compat=1.3}
\usetikzlibrary{calc}
\usepackage{colortbl}
\usepackage{siunitx}
\usepackage{pgfplots}
\usepackage{pgfplotstable}
\usepackage{tikz}

\setbeamertemplate{footline}
                  {
                    \leavevmode%
                    \hbox{%
                      \begin{beamercolorbox}[wd=.4\paperwidth,ht=2.25ex,dp=2ex,center]{author in head/foot}%
                        \usebeamerfont{author in head/foot}\insertshortauthor
                      \end{beamercolorbox}%
                      \begin{beamercolorbox}[wd=.6\paperwidth,ht=2.25ex,dp=2ex,center]{title in head/foot}%
                        \usebeamerfont{title in head/foot}\insertshorttitle\hspace*{3em}
                        %                        \insertframenumber{} / \inserttotalframenumber\hspace*{1ex}
                    \end{beamercolorbox}}%
                    \vskip0pt%
                  }

\usecolortheme{dove}
\beamertemplatenavigationsymbolsempty




%\author{Grégoire Pichon, Bora Uçar, Frédéric Vivien}
\author{Suraj Kumar}

\title[CR12 (\url{https://surakuma.github.io/courses/daamtc.html})]{}
%\title[CR12]{}
\usepackage{tikz}
\usetikzlibrary{calc}


\definecolor{pastelviolet}{rgb}{0.8, 0.6, 0.79}
\definecolor{babyblueeyes}{rgb}{0.63, 0.79, 0.95}
\definecolor{pastelyellow}{rgb}{0.99, 0.99, 0.59}
\definecolor{pastelgreen1}{rgb}{0.47, 0.87, 0.47}
\definecolor{pastelgreen}{rgb}{0, 1, 0}
\definecolor{pastelred}{rgb}{1.0, 0.41, 0.38}
\colorlet{patternblue}{blue!60}

\newcommand{\backupbegin}{
\newcounter{framenumberappendix}
\setcounter{framenumberappendix}{\value{framenumber}}
%\setbeamertemplate{footline}{}
}
\newcommand{\backupend}{
\addtocounter{framenumberappendix}{-\value{framenumber}}
\addtocounter{framenumber}{\value{framenumberappendix}}
\setbeamertemplate{footline}{
\vspace{-1cm}\small{\insertframenumber/\inserttotalframenumber}
}
}

\setbeamertemplate{caption}{\raggedright\insertcaption\par}
\makeatother

\begin{document}

\begin{frame}{{\textbf{CR12: Data-aware algorithms for matrix and tensor computations}}}

\vfill

\begin{minipage}{0.45\linewidth}
\begin{small}
\begin{minipage}[l]{0.4\linewidth}
	\vspace*{0.6cm}
		\begin{center}
		\begin{figure}
			\begin{tikzpicture}[scale=0.5, every node/.style={transform shape}]
%			\tikzstyle{taskmemory}=[draw=black, minimum height=10mm, minimum width=21mm, fill=blue!40, text=black]
			\tikzstyle{taskmemory}=[draw=black, minimum height=8mm, minimum width=25mm, fill=blue!20, text=black]
			\tikzstyle{taskcompute}=[draw=black, minimum height=16mm, minimum width=16mm, fill=none, text=black, below]
			
			\node (t0) at (0,4) [taskcompute] {}; 
			\node (t1) at (0,0) [taskmemory] {$DRAM$};

			
			\draw [<->, line width=3, orange] (t0) -- (t1);
			
			
			\node (td0)  at (t0.south) [above, scale=0.6] {$Cache$};
%			\node (td1) [above, scale=0.6] at (t1.south) {$DRAM$};
%			\node (td2) [above, scale=0.6] at (t2.south) {$DRAM$};
%			\node (td3) [above, scale=0.6] at (t3.south) {$DRAM$};
			
			\node [above] at (td0.north) {$CPU$};
			

%		\path (0, -2) -- (0.1, -2);
%		\path (0, 6) -- (0.1, 6);

			\end{tikzpicture}
			\caption{\scriptsize Sequential machine}
		\end{figure}
	\end{center}
\end{minipage}
\hfill
\begin{minipage}{0.45\linewidth}
	\begin{center}
		\begin{figure}
		\begin{tikzpicture}[scale=0.5, every node/.style={transform shape}]
		%%\tikzstyle{taskmemory}=[draw=black, minimum height=18mm, minimum width=18mm, fill=blue!40, text=black]
		\tikzstyle{taskcompute}=[draw=black, minimum height=16mm, minimum width=16mm, fill=none, text=black, below]
		
		\node (t0) at (0,0) [taskcompute] {}; 
		\node (t1) at (4,0) [taskcompute] {};
		\node (t2) at (4,4) [taskcompute] {};
		\node (t3) at (0,4) [taskcompute] {};
		
		\draw [<->, line width=3, orange] (t0) -- (t1);
		\draw [<->, line width=3, orange] (t1) -- (t2);
		\draw [<->, line width=3, orange] (t2) -- (t3);
		\draw [<->, line width=3, orange] (t3) -- (t0);
		
		%%\draw [<->, line width=3, orange] (t0) -- (t2);
		%%\draw [<->, line width=3, orange] (t1) -- (t3);
		
		\node (td0)  at (t0.south) [above, scale=0.6] {$DRAM$};
		\node (td1) [above, scale=0.6] at (t1.south) {$DRAM$};
		\node (td2) [above, scale=0.6] at (t2.south) {$DRAM$};
		\node (td3) [above, scale=0.6] at (t3.south) {$DRAM$};
		
		\node [above] at (td0.north) {$CPU$};
		\node [above] at (td1.north) {$CPU$};
		\node [above] at (td2.north) {$CPU$};
		\node [above] at (td3.north) {$CPU$};
		
%		\path (0, -2) -- (0.1, -2);
%		\path (0, 6) -- (0.1, 6);
		\end{tikzpicture}
		\caption{\scriptsize Parallel machine}
		\end{figure}
	\end{center}
\end{minipage}
\end{small}
\end{minipage}\hfill
\fcolorbox{white}{blue!5}{%
	\begin{minipage}[r]{\dimexpr0.405\linewidth-2\fboxrule-2\fboxsep\relax}
			\centering{\color{blue!70}{\small Tensors ($N$-dimensional arrays)}}
				\begin{center}
					\begin{tikzpicture}[scale=0.115, every node/.style={transform shape}]
					\pgfmathsetmacro{\rectx}{4}
					\pgfmathsetmacro{\recty}{0.5}
					\draw[blue,fill=pastelgreen] (0,0) -- node [below, scale=6, black] {Vector}++(-\rectx,0) -- ++(0,\recty) -- ++(\rectx, 0) -- cycle;
					\end{tikzpicture}$\;$
					\begin{tikzpicture}[scale=0.115, every node/.style={transform shape}]
					\pgfmathsetmacro{\rectx}{4}
					\pgfmathsetmacro{\recty}{4}
					\draw[blue,fill=pastelgreen] (0,0) -- node [below, scale=6, black] {Matrix}++(-\rectx,0) -- ++(0,\recty) -- ++(\rectx, 0) -- cycle;
					%%\addvmargin{4};
					\end{tikzpicture}$\;$
					\begin{tikzpicture}[scale=0.115, every node/.style={transform shape}]
					\pgfmathsetmacro{\cubex}{4}
					\pgfmathsetmacro{\cubey}{4}
					\pgfmathsetmacro{\cubez}{4}
					\draw[blue,fill=pastelgreen] (0,0,0) -- ++(-\cubex,0,0) -- ++(0,-\cubey,0) --node [below, scale=6, black] {3-dimensional tensor} ++(\cubex,0,0) -- cycle;
					\draw[blue,fill=pastelgreen] (0,0,0) -- ++(0,0,-\cubez) -- ++(0,-\cubey,0) -- ++(0,0,\cubez) -- cycle;
					\draw[blue,fill=pastelgreen] (0,0,0) -- ++(-\cubex,0,0) -- ++(0,0,-\cubez) -- ++(\cubex,0,0) -- cycle;
					\end{tikzpicture}$\;$
					%%	\end{center}
					%%	\begin{center}
					\end{center}
				
				\vspace*{0.1cm}
				\begin{center}
				\begin{tikzpicture}[scale=0.115, every node/.style={transform shape}]
				\pgfmathsetmacro{\cubex}{4}
				\pgfmathsetmacro{\cubey}{4}
				\pgfmathsetmacro{\cubez}{4}
				\draw[blue,fill=pastelgreen] (0,0,0) -- ++(-\cubex,0,0) -- ++(0,-\cubey,0) -- ++(\cubex,0,0) -- cycle;
				\draw[blue,fill=pastelgreen] (0,0,0) -- ++(0,0,-\cubez) -- ++(0,-\cubey,0) -- ++(0,0,\cubez) -- cycle;
				\draw[blue,fill=pastelgreen] (0,0,0) -- ++(-\cubex,0,0) -- ++(0,0,-\cubez) -- ++(\cubex,0,0) -- cycle;
				
				\draw[blue,fill=pastelgreen] (\cubex +2,0,0) -- ++(-\cubex,0,0) -- ++(0,-\cubey,0) -- ++(\cubex,0,0) -- cycle;
				\draw[blue,fill=pastelgreen] (\cubex +2,0,0) -- ++(0,0,-\cubez) -- ++(0,-\cubey,0) -- ++(0,0,\cubez) -- cycle;
				\draw[blue,fill=pastelgreen] (\cubex +2,0,0) -- ++(-\cubex,0,0) -- ++(0,0,-\cubez) -- ++(\cubex,0,0) -- cycle;
				
				\draw[blue,fill=pastelgreen] (\cubex +2 + \cubex +2,0,0) -- ++(-\cubex,0,0) -- ++(0,-\cubey,0) -- ++(\cubex,0,0) -- cycle;
				\draw[blue,fill=pastelgreen] (\cubex +2 + \cubex +2,0,0) -- ++(0,0,-\cubez) -- ++(0,-\cubey,0) -- ++(0,0,\cubez) -- cycle;
				\draw[blue,fill=pastelgreen] (\cubex +2 + \cubex +2,0,0) -- ++(-\cubex,0,0) -- ++(0,0,-\cubez) -- ++(\cubex,0,0) -- cycle;
				
				\draw[blue, fill=none] (-\cubex -1, 2.5, 0) -- ++(0, -\cubey -3.5, 0) --node [below, scale=6, black] {4-dimensional tensor} ++(\cubex +2 + \cubex +2 + \cubex + \cubex,0,0) -- ++(0, \cubey +3.5, 0) -- cycle; 
				
				%%\node [scale=2] at (0, -8) {$hello$};
				\end{tikzpicture}
			\end{center}

\end{minipage}}
%\begin{minipage}[r]{0.4\linewidth}
%	\begin{block}{\color{blue!70}{\small Tensors ($N$-dimensional arrays)}}
%%		\colorbox{red}{
%	\begin{center}
%		\begin{tikzpicture}[scale=0.115, every node/.style={transform shape}]
%		\pgfmathsetmacro{\rectx}{4}
%		\pgfmathsetmacro{\recty}{0.5}
%		\draw[blue,fill=pastelgreen] (0,0) -- node [below, scale=6, black] {Vector}++(-\rectx,0) -- ++(0,\recty) -- ++(\rectx, 0) -- cycle;
%		\end{tikzpicture}$\;$
%		\begin{tikzpicture}[scale=0.115, every node/.style={transform shape}]
%		\pgfmathsetmacro{\rectx}{4}
%		\pgfmathsetmacro{\recty}{4}
%		\draw[blue,fill=pastelgreen] (0,0) -- node [below, scale=6, black] {Matrix}++(-\rectx,0) -- ++(0,\recty) -- ++(\rectx, 0) -- cycle;
%		%%\addvmargin{4};
%		\end{tikzpicture}$\;$
%		\begin{tikzpicture}[scale=0.115, every node/.style={transform shape}]
%		\pgfmathsetmacro{\cubex}{4}
%		\pgfmathsetmacro{\cubey}{4}
%		\pgfmathsetmacro{\cubez}{4}
%		\draw[blue,fill=pastelgreen] (0,0,0) -- ++(-\cubex,0,0) -- ++(0,-\cubey,0) --node [below, scale=6, black] {3-dimensional tensor} ++(\cubex,0,0) -- cycle;
%		\draw[blue,fill=pastelgreen] (0,0,0) -- ++(0,0,-\cubez) -- ++(0,-\cubey,0) -- ++(0,0,\cubez) -- cycle;
%		\draw[blue,fill=pastelgreen] (0,0,0) -- ++(-\cubex,0,0) -- ++(0,0,-\cubez) -- ++(\cubex,0,0) -- cycle;
%		\end{tikzpicture}$\;$
%		%%	\end{center}
%		%%	\begin{center}
%		\end{center}
%	
%	\vspace*{0.1cm}
%%	}
%	
%		\begin{center}
%		\begin{tikzpicture}[scale=0.115, every node/.style={transform shape}]
%		\pgfmathsetmacro{\cubex}{4}
%		\pgfmathsetmacro{\cubey}{4}
%		\pgfmathsetmacro{\cubez}{4}
%		\draw[blue,fill=pastelgreen] (0,0,0) -- ++(-\cubex,0,0) -- ++(0,-\cubey,0) -- ++(\cubex,0,0) -- cycle;
%		\draw[blue,fill=pastelgreen] (0,0,0) -- ++(0,0,-\cubez) -- ++(0,-\cubey,0) -- ++(0,0,\cubez) -- cycle;
%		\draw[blue,fill=pastelgreen] (0,0,0) -- ++(-\cubex,0,0) -- ++(0,0,-\cubez) -- ++(\cubex,0,0) -- cycle;
%		
%		\draw[blue,fill=pastelgreen] (\cubex +2,0,0) -- ++(-\cubex,0,0) -- ++(0,-\cubey,0) -- ++(\cubex,0,0) -- cycle;
%		\draw[blue,fill=pastelgreen] (\cubex +2,0,0) -- ++(0,0,-\cubez) -- ++(0,-\cubey,0) -- ++(0,0,\cubez) -- cycle;
%		\draw[blue,fill=pastelgreen] (\cubex +2,0,0) -- ++(-\cubex,0,0) -- ++(0,0,-\cubez) -- ++(\cubex,0,0) -- cycle;
%		
%		\draw[blue,fill=pastelgreen] (\cubex +2 + \cubex +2,0,0) -- ++(-\cubex,0,0) -- ++(0,-\cubey,0) -- ++(\cubex,0,0) -- cycle;
%		\draw[blue,fill=pastelgreen] (\cubex +2 + \cubex +2,0,0) -- ++(0,0,-\cubez) -- ++(0,-\cubey,0) -- ++(0,0,\cubez) -- cycle;
%		\draw[blue,fill=pastelgreen] (\cubex +2 + \cubex +2,0,0) -- ++(-\cubex,0,0) -- ++(0,0,-\cubez) -- ++(\cubex,0,0) -- cycle;
%		
%		\draw[blue, fill=none] (-\cubex -1, 2.5, 0) -- ++(0, -\cubey -3.5, 0) --node [below, scale=6, black] {4-dimensional tensor} ++(\cubex +2 + \cubex +2 + \cubex + \cubex,0,0) -- ++(0, \cubey +3.5, 0) -- cycle; 
%		
%		%%\node [scale=2] at (0, -8) {$hello$};
%		\end{tikzpicture}
%	\end{center}
%	\end{block}
%\end{minipage}

\vfill

\begin{small}
  \begin{columns}
  \begin{column}{0.65\textwidth}
  \begin{block}{{\color{blue}{Contents}}}
    \begin{itemize}
    	\item Minimum data transfer costs of computations
    	\item Matrix and tensor computation algorithms
    	\item Popular methods to work with large tensors
%    \item Graph algorithms
%    \item Low-rank compression for sparse direct solvers
%    \item Scheduling under memory constraints
    \end{itemize}
  \end{block}
    \end{column}
  \begin{column}{0.35\textwidth}
  \begin{block}{{\color{blue}{Evaluation}}}
    \begin{itemize}
    \item Homework assignments
   \item Report/Presentation 
    \end{itemize}
  \end{block}
    \end{column}
    \end{columns}
    \end{small}
\end{frame}

\vfill  
\end{document}
