%%
%% This is file `sample-sigconf.tex',
%% generated with the docstrip utility.
%%
%% The original source files were:
%%
%% samples.dtx  (with options: `all,proceedings,bibtex,sigconf')
%% 
%% IMPORTANT NOTICE:
%% 
%% For the copyright see the source file.
%% 
%% Any modified versions of this file must be renamed
%% with new filenames distinct from sample-sigconf.tex.
%% 
%% For distribution of the original source see the terms
%% for copying and modification in the file samples.dtx.
%% 
%% This generated file may be distributed as long as the
%% original source files, as listed above, are part of the
%% same distribution. (The sources need not necessarily be
%% in the same archive or directory.)
%%
%%
%% Commands for TeXCount
%TC:macro \cite [option:text,text]
%TC:macro \citep [option:text,text]
%TC:macro \citet [option:text,text]
%TC:envir table 0 1
%TC:envir table* 0 1
%TC:envir tabular [ignore] word
%TC:envir displaymath 0 word
%TC:envir math 0 word
%TC:envir comment 0 0
%%
%% The first command in your LaTeX source must be the \documentclass
%% command.
%%
%% For submission and review of your manuscript please change the
%% command to \documentclass[manuscript, screen, review]{acmart}.
%%
%% When submitting camera ready or to TAPS, please change the command
%% to \documentclass[sigconf]{acmart} or whichever template is required
%% for your publication.
%%
%%
\documentclass[sigconf]{acmart}
%%
%% \BibTeX command to typeset BibTeX logo in the docs
\AtBeginDocument{%
  \providecommand\BibTeX{{%
    Bib\TeX}}}


\let\Bbbk\relax

\usepackage{amsfonts,amssymb}
\usepackage[mathscr]{eucal}
\usepackage{graphicx}
\usepackage{color}
\usepackage{todonotes}
\usepackage{algorithm}
\usepackage[noend]{algpseudocode}
\usepackage{xspace}
\usepackage{etoolbox}
\usepackage{xcolor}
\usepackage{multirow}
\usepackage{enumitem}
\usepackage[normalem]{ulem}
\usepackage{bm}
\usepackage{subcaption}
\usepackage{placeins}
\usepackage{balance}

\usepackage{mathtools}
\DeclarePairedDelimiter{\ceil}{\lceil}{\rceil}
\DeclarePairedDelimiter{\floor}{\lfloor}{\rfloor}
\newtheorem{definition}{Definition}
\newtheorem{lemma}{Lemma}
\newtheorem{theorem}{Theorem}
\newtheorem{fact}{Fact}

\usepackage{tikz,pgfplots}
\usetikzlibrary{matrix, decorations, patterns, positioning, shapes, 3d, calc, intersections, arrows, fit, pgfplots.fillbetween}
\usepackage{tikz-3dplot}

\usepackage{cleveref}
\Crefname{definition}{Definition}{Definitions}
\crefname{definition}{Def.}{Defs.}
\Crefname{theorem}{Theorem}{Theorems}
\crefname{theorem}{Theorem}{Theorems}
\Crefname{lemma}{Lemma}{Lemmas}
\crefname{lemma}{Lemma}{Lemmas}
\Crefname{section}{Section}{Sections}
\crefname{section}{\S}{\S}
\crefname{algorithm}{Alg.}{Algs.}
\Crefname{algorithm}{Algorithm}{Algorithms}
\crefname{figure}{Fig.}{Figs.}
\Crefname{figure}{Figure}{Figures}
\crefname{table}{Tab.}{Tabs.}
\Crefname{table}{Table}{Tables}



\newcommand{\Real}{\mathbb{R}}
\newcommand{\vc}[1]{\bm{#1}}
\newcommand{\frobenius}[1]{\left\| #1 \right\|}


% vector
\newcommand{\V}[2][]{{\bm{#1\mathbf{\MakeLowercase{#2}}}}}
% matrix
\newcommand{\M}[2][]{{\bm{#1\mathbf{\MakeUppercase{#2}}}}}
% tensor
\newcommand{\T}[2][]{\boldsymbol{#1\mathscr{\MakeUppercase{#2}}}}


\newcommand{\sttsv}{STTSV\xspace}
\newcommand{\aaomttkrp}{All-at-Once MTTKRP\xspace}
\newcommand{\arraymul}{\emph{ternary multiplication}\xspace}
\newcommand{\pluseq}{\mathrel{+}=}


\newcommand{\GB}[1]{\textcolor{red}{\textbf{GB:} #1}}
\newcommand{\KR}[1]{\textcolor{blue}{\textbf{KR:} #1}}
\newcommand{\SK}[1]{\textcolor{orange}{\textbf{SK:} #1}}



\graphicspath{{Figs/}{./}} % Specifies where to look for included images (trailing slash required)


%% Rights management information.  This information is sent to you
%% when you complete the rights form.  These commands have SAMPLE
%% values in them; it is your responsibility as an author to replace
%% the commands and values with those provided to you when you
%% complete the rights form.
\setcopyright{acmlicensed}
\copyrightyear{2026}
\acmYear{2026}
\acmDOI{XXXXXXX.XXXXXXX}
%% These commands are for a PROCEEDINGS abstract or paper.
\acmConference[SPAA '26]{Communication Lower Bounds for Symmetric Matricized-Tensor Times Khatri-Rao Product}{July 06--10,
  2026}{London, UK}
%%
%%  Uncomment \acmBooktitle if the title of the proceedings is different
%%  from ``Proceedings of ...''!
%%
%%\acmBooktitle{Woodstock '18: ACM Symposium on Neural Gaze Detection,
%%  June 03--05, 2018, Woodstock, NY}
\acmISBN{978-1-4503-XXXX-X/2018/06}


%%
%% Submission ID.
%% Use this when submitting an article to a sponsored event. You'll
%% receive a unique submission ID from the organizers
%% of the event, and this ID should be used as the parameter to this command.
%%\acmSubmissionID{123-A56-BU3}

%%
%% For managing citations, it is recommended to use bibliography
%% files in BibTeX format.
%%
%% You can then either use BibTeX with the ACM-Reference-Format style,
%% or BibLaTeX with the acmnumeric or acmauthoryear sytles, that include
%% support for advanced citation of software artefact from the
%% biblatex-software package, also separately available on CTAN.
%%
%% Look at the sample-*-biblatex.tex files for templates showcasing
%% the biblatex styles.
%%

%%
%% The majority of ACM publications use numbered citations and
%% references.  The command \citestyle{authoryear} switches to the
%% "author year" style.
%%
%% If you are preparing content for an event
%% sponsored by ACM SIGGRAPH, you must use the "author year" style of
%% citations and references.
%% Uncommenting
%% the next command will enable that style.
%%\citestyle{acmauthoryear}


%%
%% end of the preamble, start of the body of the document source.
\begin{document}

%%
%% The "title" command has an optional parameter,
%% allowing the author to define a "short title" to be used in page headers.
\title[Communication Lower Bounds for Symmetric MTTKRP]{Brief Announcement: Communication Lower Bounds for Symmetric Matricized-Tensor Times Khatri-Rao Product}

%%
%% The "author" command and its associated commands are used to define
%% the authors and their affiliations.
%% Of note is the shared affiliation of the first two authors, and the
%% "authornote" and "authornotemark" commands
%% used to denote shared contribution to the research.
%\author{Ben Trovato}
%\authornote{Both authors contributed equally to this research.}
%\email{trovato@corporation.com}
%\orcid{1234-5678-9012}
%\author{G.K.M. Tobin}
%\authornotemark[1]
%\email{webmaster@marysville-ohio.com}
%\affiliation{%
%  \institution{Institute for Clarity in Documentation}
%  \city{Dublin}
%  \state{Ohio}
%  \country{USA}
%}
%
%\author{Lars Th{\o}rv{\"a}ld}
%\affiliation{%
%  \institution{The Th{\o}rv{\"a}ld Group}
%  \city{Hekla}
%  \country{Iceland}}
%\email{larst@affiliation.org}

%%
%% By default, the full list of authors will be used in the page
%% headers. Often, this list is too long, and will overlap
%% other information printed in the page headers. This command allows
%% the author to define a more concise list
%% of authors' names for this purpose.
\renewcommand{\shortauthors}{Author1 et al.}

%%
%% The abstract is a short summary of the work to be presented in the
%% article.
\begin{abstract}
TODO....


%Al Daas et al.~\cite{} 

Our earlier conference article~\cite{} studied the parallel communication cost of multiplying the same vector along two modes of a $3$-dimensional symmetric tensor. This can be viewed as a special case of the present work where the CP rank is $1$.



\end{abstract}

%%
%% The code below is generated by the tool at http://dl.acm.org/ccs.cfm.
%% Please copy and paste the code instead of the example below.
%%
%\begin{CCSXML}
%<ccs2012>
% <concept>
%  <concept_id>00000000.0000000.0000000</concept_id>
%  <concept_desc>Do Not Use This Code, Generate the Correct Terms for Your Paper</concept_desc>
%  <concept_significance>500</concept_significance>
% </concept>
% <concept>
%  <concept_id>00000000.00000000.00000000</concept_id>
%  <concept_desc>Do Not Use This Code, Generate the Correct Terms for Your Paper</concept_desc>
%  <concept_significance>300</concept_significance>
% </concept>
% <concept>
%  <concept_id>00000000.00000000.00000000</concept_id>
%  <concept_desc>Do Not Use This Code, Generate the Correct Terms for Your Paper</concept_desc>
%  <concept_significance>100</concept_significance>
% </concept>
% <concept>
%  <concept_id>00000000.00000000.00000000</concept_id>
%  <concept_desc>Do Not Use This Code, Generate the Correct Terms for Your Paper</concept_desc>
%  <concept_significance>100</concept_significance>
% </concept>
%</ccs2012>
%\end{CCSXML}
%
%\ccsdesc[500]{Do Not Use This Code~Generate the Correct Terms for Your Paper}
%\ccsdesc[300]{Do Not Use This Code~Generate the Correct Terms for Your Paper}
%\ccsdesc{Do Not Use This Code~Generate the Correct Terms for Your Paper}
%\ccsdesc[100]{Do Not Use This Code~Generate the Correct Terms for Your Paper}


\keywords{CP Tensor decomposition, MTTKRP computation, Communication lower bounds, Parallel algorithms}



%\received{20 February 2007}
%\received[revised]{12 March 2009}
%\received[accepted]{5 June 2009}

%%
%% This command processes the author and affiliation and title
%% information and builds the first part of the formatted document.
\maketitle

\section{Introduction}
\label{sec:intro}
An $R$-rank symmetric Canonical Polyadic (CP) decomposition of a $3$-dimensional tensor $\T{X}$ expresses it as a sum of $R$ symmetric rank-one tensors. More precisely, there is a matrix $\M{U}$ whose $r$th column is denoted by $\V{u}_r$, such that $\T{X} \approx \sum_{r=1}^R \V{u}_r \circ \V{u}_{r} \circ \V{u}_{r}$, where $\circ$ denotes an outer product \cite{Kolda15}. \Cref{alg:cpals} specifies a method to compute symmetric CP decomposition of a $3$-dimensional tensor. $\M{X_{(1)}}(U\odot U)$ is the bottleneck computation of this algorithm and known as the symmetric Matricized-Tensor Times Khatri-Rao Product (MTTKRP) in the literature. Here $\M{X_{(1)}}$ denotes the first unfolding matrix of $\T{X}$ and $\odot$ denotes Khatri-Rao product (defined in the next section). The goal of this work is to determine how much data movement is required to perform the $3$-dimensional symmetric MTTKRP in parallel.

\begin{algorithm}
	\caption{CP-ALS method to compute symmetric CP decomposition\label{alg:cpals}}
	\begin{algorithmic}[1]
		\Require $\T{X}\in \Real^{n\times n\times n}$ is a symmetric 3-dimensional tensor, initial factor matrix $\M{X}\in \Real^{n\times R}$
		\Ensure $\big(\lambda \in \Real^R, \M{U} \in \Real^{n\times R}\big)$ : A rank-$R$ CP decomposition of $\T{X}$
		\State Initialize each column of $\M{U}$ to a random unit vector (or based on some heuristics)
		\Repeat
		\State $\M{G} = (\M{U}^T\M{U}) \ast (\M{U}^T\M{U})$ \Comment{$\ast$ denotes the elementwise product}
		\State $\M{V} = \M{X_{(1)}}(U\odot U)$
		\State $\M{U} = \M{V}\M{G}^\dagger$ \Comment $G^\dagger$ is the pseudo inverse of $G$
		\State Normalize columns of $\M{U}$ and store norms in $\lambda$
		\Until{convergence or the maximum number of iterations}
	\end{algorithmic}
\end{algorithm}








\section{Related Work}
\label{sec:relatedwork}
TODO..



\section{Notations and Preliminaries}
\label{sec:notations}


Now we present our notations. $\T{X}$ denotes a $3$-dimensional symmetric tensor of dimensions $n\times n\times n$. $\M{X}_{(i)}$ represents $i$th unfolding matrix of $\T{X}$ of dimensions $n \times n^2$. $\M{U}\in \Real^{n\times R}$ is the factor matrix associated with a $R$-rank CP decomposition of $\T{X}$. $\M{U}_{ir}$ denotes the $i$th row and $r$th column element of $\M{U}$. Similarly, $\T{X}_{ijk}$ denotes the element at index $(i,j,k)$ of $\T{X}$. As $\T{X}$ is symmetric, therefore we have $\M{X}_{(1)} = \M{X}_{(2)} = \M{X}_{(3)}$. We also have same $\T{X}_{ijk}$ for all permutations of $i$, $j$ and $k$, i.e., $\T{X}_{ijk} =\T{X}_{ikj}=\T{X}_{jik}=\T{X}_{jki}=\T{X}_{kij}=\T{X}_{kji}$.  $\M{A}_{ir}\odot \M{B}_{jr}$ denotes Khatri-Rao Product of matrices $\M{A} \in \Real^{n_1\times R}$ and $\M{B} \in \Real^{n_2\times R}$, and has dimensions $n_1n_2 \times R$. The $(i\cdot n_2 +j, r)$th element of $\M{A} \odot \M{B}$ is computed as $\M{A}_{ir}\cdot \M{B}_{jr}$. The CP rank of a cubical tensor of dimensions $n\times n\times n$ is at most $n^2$~\cite{Kruskal89,LandsbergBook}. Therefore, we also have $1\leq R\leq n^2$. 


%\todo[inline]{Also find another reference for rank bound.}

We present a direct way to perform symmetric MTTKRP in \Cref{alg:symmMTTKRP:naive}. It does not take symmetry into account. We call the operation $\T{X}_{ijk}\cdot \M{U}_{jr} \cdot \M{U}_{kr}$ as \emph{\arraymul}. The algorithm performs a total of $n^3R$ \emph{\arraymul} operations.


\Cref{alg:symmMTTKRP:symmetry} exploits the symmetry of the tensor and works only with the lower tetrahedral portion of $\T{X}$, $\T{X}_{ijk}$ where $i\geq j\geq k$. The total number of points in the iteration space is $n(n+1)(n+2)R/6$ of which $n(n-1)(n-2)R/6$ points correspond to the strict lower tetrahedral portion of $\T{X}$, $\T{X}_{ijk}$ where $i> j> k$. \Cref{alg:symmMTTKRP:symmetry} performs a total of $n^2(n+1)R/2$ \emph{\arraymul} operations, approximately half the number of those in \Cref{alg:symmMTTKRP:naive}.



\begin{algorithm}[htb]
	\caption{\label{alg:symmMTTKRP:naive}Pseudocode of symmetric MTTKRP}
	\begin{algorithmic}
		\Require $\T{X}\in \Real^{n\times n\times n}$ is a symmetric tensor and $\M{U}\in \Real^{n\times R}$ is a matrix.
		\Ensure $\M{V}\in \Real^{n\times R}$ such that  $\M{V} = \M{X_{(1)}} (\M{U}\odot \M{U})$.
		\State Initialize vector $\M{V}$ to $0$
		\For{$i=1 \text{ to } n$}
		\For{$j=1 \text{ to } n$}
		\For{$k=1 \text{ to } n$}
		\For{$r=1 \text{ to } R$}
		\State $\M{V}_{ir} \pluseq \T{X}_{ijk}\cdot \M{U}_{jr} \cdot \M{U}_{kr}$
		\EndFor
		\EndFor
		\EndFor
		\EndFor
	\end{algorithmic}
\end{algorithm}

\begin{algorithm}[htb]
	\caption{\label{alg:symmMTTKRP:symmetry}Symmetric MTTKRP that exploits symmetry}
	\begin{algorithmic}
		\Require $\T{X}\in \Real^{n\times n\times n}$ is a symmetric tensor and $\M{U}\in \Real^{n\times R}$ is a matrix.
		\Ensure $\M{V}\in \Real^{n\times R}$ such that  $\M{V} = \M{X_{(1)}} (\M{U}\odot \M{U})$.
		\State Initialize vector $\M{V}$ to $0$
		\For{$i=1 \text{ to } n$}
		\For{$j=1 \text{ to } i$}
		\For{$k=1 \text{ to } j$}
		\For{$r=1 \text{ to } R$}
		\State $\M{V}_{ir} \pluseq \T{X}_{ijk}\cdot \M{U}_{jr} \cdot \M{U}_{kr}$
		\If{$i\neq j$ and $j\neq k$}
		\State $\M{V}_{ir} \pluseq 2\T{X}_{ijk}\cdot \M{U}_{jr} \cdot \M{U}_{kr}$
		\State $\M{V}_{jr} \pluseq 2\T{X}_{ijk}\cdot \M{U}_{ir} \cdot \M{U}_{kr}$
		\State $\M{V}_{kr} \pluseq 2\T{X}_{ijk}\cdot \M{U}_{ir} \cdot \M{U}_{jr}$
		\ElsIf{$i==j$ and $j\neq k$}
		\State $\M{V}_{ir} \pluseq 2\T{X}_{ijk}\cdot \M{U}_{jr} \cdot \M{U}_{kr}$
		\State $\M{V}_{kr} \pluseq \T{X}_{ijk}\cdot \M{U}_{ir} \cdot \M{U}_{jr}$
		\ElsIf{$i\neq j$ and $j==k$}
		\State $\M{V}_{ir} \pluseq \T{X}_{ijk}\cdot \M{U}_{jr} \cdot \M{U}_{kr}$
		\State $\M{V}_{jr} \pluseq 2\T{X}_{ijk}\cdot \M{U}_{ir} \cdot \M{U}_{kr}$
		\Else
		\State $\M{V}_{ir} \pluseq \T{X}_{ijk}\cdot \M{U}_{jr} \cdot \M{U}_{kr}$
		\EndIf
		\EndFor
		\EndFor
		\EndFor
		\EndFor
	\end{algorithmic}
\end{algorithm}



A parallel atomic symmetric MTTKRP algorithm computes each \emph{\arraymul} on one processor. This implies that while computing a single \arraymul all $3$ inputs are accessed on that processor. This assumption is necessary for our communication lower bounds. We focus only on parallel atomic algorithms for symmetric MTTKRP that perform $n^2(n+1)/2$ \arraymul operations. It is reasonable for an algorithm to break our assumption as it restricts partial reuse across processors. For instance, one way to compute $\M{X_{(1)}} (\M{U}\odot \M{U})$ is to first compute $\M{U}\odot \M{U}$ and then perform a matrix multiplication of $\M{X_{(1)}}$ and the result. Minimum data transfer cost of this approach is $\mathcal{O}(n^2R/P)^{2/3}$~\cite{ABGKR22}. We show in \Cref{sec:symmMTTKRP} that the cost of our approach is asymptotically better.  

\subsection{Computation Model}
\label{sec:model}

\section{Fundamental Results}
\label{sec:dundamental}
We relate the amount of computations to data accesses in order to determine the minimum amount of data a processor must access to perform its local computations.


We obtain the following lemma from \cite[Lemma 4.1]{BKR18} by fixing the number of dimensions. We consider the projections based on the indices of the tensor and matrices used in \Cref{alg:symmMTTKRP:naive}. 

\begin{lemma}
	\label{lem:basicHBL}
	Let $V$ be a finite set of points in $\mathbb{Z}^4$. Let $\phi_{ijk}(V)$ be the projection of $V$ in $i-j-k$ space, i.e., all points $(i,j,k)$ such that there exists a $r$ so that $(i,j,k,r) \in V$. Let $\phi_{ir}(V)$ be the projection of $V$ in $i-r$ plane, i.e., all points $(i,r)$ such that there exists a $(j,k)$ so that $(i,j,k,r) \in V$.  Define $\phi_{jr}(V)$ and $\phi_{kr}(V)$ similarly. 
	Then 
	$$|V| \leq |\phi_{ijk}(V)|^{2/3} \cdot |\phi_{ir}(V)|^{1/3}\cdot |\phi_{jr}(V)|^{1/3}\cdot |\phi_{kr}(V)|^{1/3}.$$
\end{lemma}

We now extend \Cref{lem:basicHBL}  for a symmetry et of points corresponding to the first three coordinates.
\begin{lemma}
	\label{lem:symmHBL}
	Let $V$ be a finite set of points in  $\{(i, j, k, r)\in\mathbb{Z}^4 \, | \, i>j>k\}$. Let $\phi_{ijk}(V)$ be the projection of $V$ in $i-j-k$ space, i.e., all points $(i,j,k)$ such that there exists a $r$ so that $(i,j,k,r) \in V$. Let $\phi_{ir}(V)$ be the projection of $V$ in $i-r$ plane, i.e., all points $(i,r)$ such that there exists a $(j,k)$ so that $(i,j,k,r) \in V$.  Define $\phi_{jr}(V)$ and $\phi_{kr}(V)$ similarly. 
	Then 
	$$6|V| \leq (6|\phi_{ijk}(V)|)^{2/3} \cdot |\phi_{ir}(V) \cup \phi_{jr}(V) \cup \phi_{kr}(V)|.$$
\end{lemma}

\begin{proof}
	TODO....
\end{proof}

\subsection{Existing Results}
\label{sec:fundamental:existingResults}

We now present some definitions and results that we use in \Cref{sec:symmMTTKRP} to analytically solve our optimization problems.

\begin{definition}[{\cite[eq. (3.20)]{BV04}}]
	\label{def:quasiconvex}
	A differentiable function $g:\Real^d\rightarrow \Real$ is \emph{quasiconvex} if its domain is a convex set and for all $\vc{x},\vc{y} \in \textbf{dom} \; g$, 
	$$g(\vc{y}) \leq g(\vc{x}) \text{ implies that } \langle \nabla g(\vc{x}), \vc{y} - \vc{x} \rangle \leq 0.$$
\end{definition}

\begin{definition}[{\cite[eq. (5.49)]{BV04}}]
	\label{def:KKT}
	Consider an optimization problem of the form 
	\begin{equation}
	\label{eq:optprob}
	\min_{\vc{x}} f(\vc{x}) \quad \text{ subject to } \quad \vc{g}(\vc{x}) \leq \vc{0}
	\end{equation} 
	where $f:\Real^d \rightarrow \Real$ and $\vc{g}:\Real^d\rightarrow \Real^c$ are both differentiable. 
	Define the dual variables $\vc{\mu}\in\mathbb{R}^c$, and let $\vc{J}_{\vc{g}}$ be the Jacobian of $\vc{g}$.
	The \emph{Karush-Kuhn-Tucker (KKT)} conditions of $(\vc{x},\vc{\mu})$ are as follows:
	\begin{itemize}
		\item \emph{Primal feasibility}: $\vc{g}(\vc{x}) \leq \vc{0}$;
		\item \emph{Dual feasibility}: $\vc{\mu} \geq 0$;
		\item \emph{Stationarity}: $\nabla f(\vc{x}) + \vc{\mu} \cdot \vc{J}_{\vc{g}}(\vc{x}) = \vc{0}$;
		\item \emph{Complementary slackness}: $\mu_i g_i(\vc{x})=0$ for all $i\in \{1,\dots,c\}$. 
	\end{itemize}
\end{definition}


\begin{lemma}[{\cite[Lemma 3]{ABGKR22}}]
	\label{lem:KKT}
	Consider an optimization problem of the form given in \cref{eq:optprob}.
	If $f$ is a convex function and each $g_i$ is a quasiconvex function, then the KKT conditions are sufficient for optimality.	
\end{lemma}

\begin{lemma}
	\label{lem:quasiconvex:x23y}
	The function $g_0(\vc{x}) = L - x^{2/3}y$, for some constant $L$, is quasiconvex in the positive quadrant.
\end{lemma}
\begin{proof}
	TODO...
\end{proof}








\section{Symmetric MTTKRP Lower Bounds}
\label{sec:symmMTTKRP}


\begin{lemma}
	\label{lem:projlb}
	Consider a parallel algorithm that performs the symmetric MTTKRP computation, $\M{V}=\M{X}_{(1)} (U\odot U)$, where $\M{U}$ and $\M{V}$ are matrices of dimensions $n \times R$ and $\M{X}_{(1)}$, which is the $1$st unfolding matrix of symmetric tensor $\T{X}$, has dimensions $n\times n^2$. Any processor that performs at least $1/P$th of the points in the iteration space involving the strict lower tetrahedron of $\T{X}$ must access at least $nR/P$ elements of each $\M{U}$ and $\M{V}$, and $n(n-1)(n-2)/6P$ elements of the strict lower tetrahedron of $\T{X}$.    
\end{lemma}



\begin{proof}
	The total number of iteration points that must be performed involving lower tetrahedron of $\T{X}$ is $n(n-1)(n-2)R/6$.
	Consider a processor that performs at least $1/P$th of these iteration points. We first focus on the number of elements of $\M{U}$ this processor must access. Each element of $\M{U}$ is involved in $(n-1)(n-2)/2$ iteration points. Three different elements of $\M{U}$ is accessed in each iteration point.
	If the processor accesses fewer than $nR/P$ elements of $\M{U}$, then it would perform fewer than $n(n-1)(n-2)R/6P$ iteration points, which is a contradiction. The same set of arguments applies for $\M{V}$. 
	Finally, each element of $\T{X}$ is involved in $R$ iteration points.
	If the processor accesses fewer than $n(n-1)(n-2)/6P$ elements of $\T{X}$, then it would perform fewer than $n(n-1)(n-2)R/6P$ iteration points, which is again a contradiction.
\end{proof}





\begin{lemma}
	\label{lem:symmMTTKRPOpt}
	Consider the following optimization problem 
	$$\min_{\V{x}\in\mathbb{R}^2} x_1 + 2x_2$$
	such that
	$x_1^2x_2^3 \geq \frac{1}{6^2}(\frac{n(n-1)(n-2)R}{P})^3,$\\ 
	$\frac{n(n-1)(n-2)}{6P} \leq x_1 \leq \frac{n(n-1)(n-2)}{6P},$\\
	$\frac{nR}{P} \leq x_2 \leq nR,$
	where $P,n$, and $R$ are all positive integers greater than or equal to 1.
	The constraints induce three scenarios for the optimal solution $\V{x}^*$ :
	\begin{itemize}
		\item If $1\leq R \leq \frac{(n-1)(n-2)}{2^3}$  and $P\leq \frac{n(n-1)(n-2)}{(2^3 R)^{3/2}}$ then $x_1^* = \frac{n(n-1)(n-2)}{6P}, x_2^* = (\frac{n(n-1)(n-2)}{P})^{1/3}R;$
		
		\item If $\frac{(n-1)(n-2)}{2^3} \leq R \leq n^2$ and $P\leq \frac{2^3nR}{((n-1)(n-2))^{3/2}}$ then $x_1^* = \frac{((n-1)(n-2))^{3/2}}{6}, x_2^* = \frac{nR}{P};$
		\item If ($1\leq R \leq \frac{(n-1)(n-2)}{2^3}$  and $P > \frac{n(n-1)(n-2)}{(2^3 R)^{3/2}}$) or ($\frac{(n-1)(n-2)}{2^3} \leq R \leq n^2$ and $P > \frac{2^3nR}{((n-1)(n-2))^{3/2}}$) then $x_1^* = \frac{2^{4/5}}{3} (\frac{n(n-1)(n-2)R}{P})^{3/5},  x_2^* =\frac{3x_1^*}{4}$.
	\end{itemize}
	This can be visualized as follows:
	\todo[inline]{complete the visualization part}
	Case I: $1\leq R \leq \frac{(n-1)(n-2)}{2^3}$\\
	Case II: $\frac{(n-1)(n-2)}{2^3} \leq R \leq n^2$
%	\begin{center}
%		\begin{tikzpicture}[scale=0.5, every node/.style={transform shape}]
%		\draw [->, thick] (-0.1,0) -- (16,0) node [below right,pastelgreen,scale=2] {$P$};
%		\draw (0, 0.1) -- node [below, pastelred, scale=2]{$1$}(0,-0.1);
%		\draw (5, 0.1) -- node [below, pastelred, scale=2]{$n_1$}(5,-0.1);
%		\draw (10, 0.1) -- node [below, pastelred, scale=2] {$\frac{n_1n_2}{r}$}(10,-0.1);
%		
%		\node[align=left,below,scale=1.5] at (2.5, -0.4) {$x_1^*=\frac{n_1n_2}{P}$\\ $x_2^*=\frac{n_1r}{P}$};
%		\node[align=left,below,scale=1.5] at (7.5, -0.6) {$x_1^*=\frac{n_1n_2}{P}$\\ $x_2^*=r$};
%		\node[align=center,below,scale=1.5] at (13.25, -0.8) {$x_1^*=x_2^*= \left(\frac{n_1n_2r}{P}\right)^{1/2}$};	
%		\end{tikzpicture}
%	\end{center}
\end{lemma}


\begin{theorem}
	\label{thm:memindeplb}
	Consider the symmetric MTTKRP computation, $\M{V}=\M{X}_{(1)} (U\odot U)$, where $\M{U}$ and $\M{V}$ are matrices of dimensions $n \times R$ and $\M{X}_{(1)}$, which is the $1$st unfolding matrix of symmetric tensor $\T{X}$, has dimensions $n\times n^2$. 
	Suppose a parallel atomic algorithm using $P$ processors begins with one copy of $\M{U}$ and the strict lower tetrahedron of $\T{X}$, and ends with one copy of $\M{V}$. 
	If each processor performs the ternary multiplications associated with $1/P$th of the points in the iteration space involving the strict lower tetrahedron of $\T{X}$, then at least one processor must communicate at least $W$ words of data where 
	$$W= \begin{cases}
	\quad 2(\frac{n(n-1)(n-2)}{P})^{1/3}R - 2\frac{nR}{P} &\text{ if } \quad 1\leq R \leq \frac{(n-1)(n-2)}{2^3}  \text{ and } P\leq \frac{n(n-1)(n-2)}{(2^3 R)^{3/2}}\\
	\quad \frac{((n-1)(n-2))^{3/2}}{6} - \frac{n(n-1)(n-2)}{6P} &\text{ if } \quad \frac{(n-1)(n-2)}{2^3} \leq R \leq n^2 \text{ and } P\leq \frac{2^3nR}{((n-1)(n-2))^{3/2}}\\
	\quad \frac{5}{2^{1/5}\cdot 3} (\frac{n(n-1)(n-2)R}{P})^{3/5}  - \frac{n(n-1)(n-2)}{6P} - 2\frac{nR}{P} &\text{ if } \quad (1\leq R \leq \frac{(n-1)(n-2)}{2^3}  \text {and } P > \frac{n(n-1)(n-2)}{(2^3 R)^{3/2}}) \text{ or } (\frac{(n-1)(n-2)}{2^3} \leq R \leq n^2 \text{ and } P > \frac{2^3nR}{((n-1)(n-2))^{3/2}})\\
	\end{cases}.$$
	
\end{theorem}


\begin{proof}
	To begin we note that there are $\frac{n(n-1)(n-2)}{6}$ elements in the strict lower tetrahedron of $\T{X}$ and $nR$ elements each in $\M{U}$ and $\M{V}$.
	There must be a processor that owns at most $1/P$th of these elements as otherwise input and output elements need to be replicated contradicting our assumptions.
	
	We determine how many elements this processor must access to perform the ternary multiplications of the assigned iteration points. Let $F$ be the set of points $(i,j,k,r)$ associated with the strict lower tetrahedron of $\T{X}$ assigned to this processor.
	Then $F\subseteq \{(i,j,k,r)\in\mathbb{Z}^4|i>j>k\}$ and $|F| = \frac{n(n-1)(n-2)R}{6P}$
	by our assumption that this processor performs $1/P$th of the iteration points.
	
	By \Cref{lem:symmHBL}, we know that 
	$$(6|\phi_{ijk}(F)|)^{2/3}|\phi_{ir}(F) \cup \phi_{jr}(F) \cup \phi_{kr}(F)| \geq 6|F| = \frac{n(n-1)(n-2)R}{P}\text.$$ 
	
	The union of projections of $F$, $\phi_{ir}(F) \cup \phi_{jr}(F) \cup \phi_{kr}(F)$, gives the indices of the elements of each matrix $\M{U}$ and $\M{V}$ accessed or modified by the processor.  Similarly, $\phi_{ijk}(F)$ gives the indices of the tensor elements that are accessed by the processor. 
	By \cref{lem:projlb}, we have lower bound constraints on $|\phi_{ir}(F) \cup \phi_{jr}(F) \cup \phi_{kr}(F)|$ and $|\phi_{ijk}(F)|$: $\frac{nR}{P} \leq |\phi_{ir}(F) \cup \phi_{jr}(F) \cup \phi_{kr}(F)|$ and $\frac{n(n-1)(n-2)}{6P} \leq |\phi_{ijk}(F)|$. Clearly a projection onto an array cannot be larger than the array itself, thus we have $|\phi_{ir}(F) \cup \phi_{jr}(F) \cup \phi_{kr}(F)| \leq nR$ and $ |\phi_{ijk}(F)| \leq \frac{n(n-1)(n-2)}{6}$.
	
	To minimize the communication we want to minimize the number of elements accessed by this processor.
	Thus we want to minimize $|\phi_{ijk}(F)| + 2|\phi_{ir}(F) \cup \phi_{jr}(F) \cup \phi_{kr}(F)|$ subject to the constraints above and the result follows by \cref{lem:symmMTTKRPOpt}.
\end{proof}


\begin{figure*}[htb]
	\includegraphics[width=0.85\linewidth]{symmetricMTTKRP.JPEG}
\end{figure*}


\section{Parallel Algorithms}
\label{sec:algorithms}
\section{Conclusion}
\label{sec:conclusion}


\begin{itemize}
	\item Future work -- $d$-dimensional computations
\end{itemize}

%\subsection{Communication lower bounds for the full computation}
%\begin{theorem}
%	\label{thm:memindeplb:simplified}
%	Consider the symmetric MTTKRP computation, $\M{V}=\M{X}_{(1)} (U\odot U)$, where $\M{U}$ and $\M{V}$ are matrices of dimensions $n \times R$ and $\M{X}_{(1)}$, which is the $1$st unfolding matrix of symmetric tensor $\T{X}$, has dimensions $n\times n^2$. 
%	Suppose a parallel atomic algorithm using $P$ processors begins with one copy of $\M{U}$ and $\T{X}$, and ends with one copy of $\M{V}$. 
%	If each processor performs the ternary multiplications associated with $1/P$th of the points in the iteration space, then at least one processor must communicate at least $W$ words of data where 
%	$$W= \begin{cases}
%	\quad 2(\frac{n(n-1)(n-2)}{P})^{1/3}R - 2\frac{nR}{P} &\text{ if } \quad 1\leq R \leq \frac{(n-1)(n-2)}{2^3}  \text{ and } P\leq \frac{n(n-1)(n-2)}{(2^3 R)^{3/2}}\\
%	\quad \frac{((n-1)(n-2))^{3/2}}{6} - \frac{n(n-1)(n-2)}{6P} &\text{ if } \quad \frac{(n-1)(n-2)}{2^3} \leq R \leq n^2 \text{ and } P\leq \frac{2^3nR}{((n-1)(n-2))^{3/2}}\\
%	\quad \frac{5}{2^{1/5}\cdot 3} (\frac{n(n-1)(n-2)R}{P})^{3/5} &\text{ if } \quad (1\leq R \leq \frac{(n-1)(n-2)}{2^3}  \text {and } P > \frac{n(n-1)(n-2)}{(2^3 R)^{3/2}}) \text{ or } (\frac{(n-1)(n-2)}{2^3} \leq R \leq n^2 \text{ and } P > \frac{2^3nR}{((n-1)(n-2))^{3/2}})\\
%	\end{cases}.$$
%	
%\end{theorem}
%
%\begin{proof}
%	\todo[inline]{TODO: for the general statement}
%\end{proof}

\newpage

\section{Cubical MTTKRP Lower Bounds}
\label{sec:cubMTTKRP}
\begin{figure*}[htb]
	\includegraphics[width=0.85\linewidth]{cubicalMTTKRP.JPEG}
\end{figure*}


%\subsection{Template Parameters}
%
%
%%\begin{itemize}
%%\item {\texttt{anonymous,review}}: Suitable for a ``double-anonymous''
%%  conference submission. Anonymizes the work and includes line
%%  numbers. Use with the \texttt{\string\acmSubmissionID} command to print the
%%  submission's unique ID on each page of the work.
%%\item{\texttt{authorversion}}: Produces a version of the work suitable
%%  for posting by the author.
%%\item{\texttt{screen}}: Produces colored hyperlinks.
%%\end{itemize}
%
%
%\section{Acknowledgments}
%\begin{acks}
%%To Robert, for the bagels and explaining CMYK and color spaces.
%\end{acks}

%%
%% The next two lines define the bibliography style to be used, and
%% the bibliography file.
\bibliographystyle{ACM-Reference-Format}
\bibliography{symmetricMTTKRP.bib}


%%%
%%% If your work has an appendix, this is the place to put it.
%\appendix
%\section{Research Methods}

\end{document}
\endinput
%%
%% End of file `sample-sigconf.tex'.
