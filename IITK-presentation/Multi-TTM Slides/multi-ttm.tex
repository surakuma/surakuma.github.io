\documentclass[aspectratio=169]{beamer}
\usetheme{Madrid}
\usepackage{todonotes}
\usepackage{graphicx}
\usepackage{color}
\usepackage{subfig}
\usepackage[noend]{algpseudocode}

\usepackage{amsmath}
\usepackage{xspace}
\usepackage{float}
\usepackage{tikz}


\usetikzlibrary{matrix,snakes, patterns, positioning, shapes, calc, intersections, arrows, fit}


\definecolor{pastelviolet}{rgb}{0.8, 0.6, 0.79}
\definecolor{babyblueeyes}{rgb}{0.63, 0.79, 0.95}
\definecolor{pastelyellow}{rgb}{0.99, 0.99, 0.59}
\definecolor{pastelgreen}{rgb}{0.47, 0.87, 0.47}
\definecolor{pastelred}{rgb}{1.0, 0.41, 0.38}
\colorlet{patternblue}{blue!60}

\graphicspath{{./diagrams/}{../expes/plots/}}


\newcommand{\heteroprio}{{HeteroPrio}\xspace}
\newcommand{\heteroprioD}{{HeteroPrioDep}\xspace}

\newcommand{\copt}{{C_{\max}^{\textsc Opt}}\xspace}
\newcommand{\HE}{C_{\max}^{\textsc HP}\xspace}

\title[HeteroPrio for Task Graphs]{Analysis of a List Scheduling Algorithm for\\ TaskGraphs on Two Types of Resources}
\author[{\sc Inria}]{Lionel {\sc Eyraud-Dubois}, 
	Inria Bordeaux, France\\
	{\it lionel.eyraud-dubois@inria.fr}\\
	 and\\ 
	 Suraj {\sc Kumar}, 
	 Inria Paris, France\\
     {\it suraj.kumar@inria.fr}
} 
\centering
%%TODO: Find a good way to achieve the same feature
\date[IPDPS 2020]{\includegraphics[width=.35\linewidth]{ipdps2020-logo.jpg}}
\begin{document}
\maketitle

%%\begin{frame}{Introduction}
%%\begin{itemize}
%%	\item Accelerators are becoming standard in HPC community
%%	\begin{itemize}
%%		\item Provide massive computational power at a limited cost 
%%		\item 144 systems in TOP 500 list use accelerators
%%	\end{itemize}
%%	\item  Exploiting the full potential of hybrid platforms is challenging
%%	\begin{itemize}
%%		\item Difficult to precisely model computation and data transfer times
%%		\item Scheduling 
%%%%		on these platforms 
%%			  is a well known hard optimization problem
%%	\end{itemize}
%%%%	\item Asynchronous task based runtime systems show promising performance on hybrid platforms.
%%	\item Task based runtime systems are getting popular to make good utilization of resources
%%			\begin{columns}
%%		\null \hfill
%%		\begin{column}{0.35\linewidth}
%%			\begin{center}
%%%%				\includegraphics[scale=0.2]{taskGraph.eps}
%%				\includegraphics[scale=0.12]{Cholesky-4.pdf}
%%			\end{center}
%%		\end{column}
%%		\begin{column}{0.65\linewidth}
%%			\begin{itemize}
%%				\item Applications can be expressed as task graphs
%%				\item Vertices represent tasks
%%				\item Edges represent dependencies among tasks
%%				\item Runtime manages scheduling of computations and communications
%%			\end{itemize}
%%		\end{column}
%%		\end{columns}
%%	 
%%\end{itemize}
%%\end{frame}

%%\begin{frame}{\heteroprio Scheduler -- Two Types of Resources}
%%%%\vspace*{-0.2cm}
%%\begin{itemize}
%%	\item Resource-centric scheduling algorithm, selects the most favorable task first  
%%        \item If no ready task remains, restart running tasks if it allows to finish them earlier
%%	\item Achieves 3.41-approximation for independent tasks (Beaumont \emph{et al.}, IPDPS 2017)
%%\end{itemize}
%%%% \begin{block}{}
%%%% %%	\begin{center}
%%%% 	\begin{columns}
%%%% 		\begin{column}[c]{.45\linewidth}
%%%% 			\vspace*{-0.5cm}
%%%% 			\begin{itemize}
%%%% 				\vfill
%%%% 				\item {\scriptsize Resource1 is idle at $t$}
%%%% 				\vfill 
%%%% 				\item {\scriptsize Resource1 selects task $A$}
%%%% 				\vfill
%%%% 				\item {\scriptsize Faster resource may spoliate tasks running on slow resource}
%%%% 				\vfill
%%%% 				\item {\scriptsize Resource2 spoliates task $A$}
%%%% 				\vfill
%%%% 			\end{itemize}
%%%% 		\end{column}
%%%% 		\begin{column}[c]{.45\linewidth}
%%%% %%			\vspace*{-0.2cm}
%%%% 			\centering \includegraphics[scale=0.35]{hp1} 
%%%% %%			\vspace*{-0.25cm}
%%%% 			\centering \includegraphics[scale=0.35]{hp2} 
%%%% %%			\vspace*{-0.2cm}
%%%% 			\centering \includegraphics[scale=0.35]{hp3} 
%%%% 		\end{column}
%%%% 	\end{columns}
%%%% %%	\end{center}
%%%% \end{block}
%%%%\vspace*{-0.25cm}
%%\begin{block}{Our Contributions}
%%	\begin{itemize}
%%%%		\vfill
%%		\item Extend \heteroprio scheduler for task graphs -- \heteroprioD
%%%%		\vfill
%%		\item Approximation proofs and worst case examples for \heteroprioD 
%%%%		\vfill
%%		\item Empirical evaluation of \heteroprioD with state-of-the-art schedulers
%%%%		\vfill
%%	\end{itemize}
%%\end{block}
%%\end{frame}
%%
%%%%\begin{frame}{Our Contributions}
%%%%\begin{itemize}
%%%%	\vfill
%%%%	\item Extend \heteroprio scheduler for task graphs -- \heteroprioD
%%%%	\vfill
%%%%	\item Approximation proofs and worst case examples for \heteroprioD scheduler
%%%%	\vfill
%%%%	\item Empirical evaluation of \heteroprioD with state-of-the-art schedulers
%%%%	\vfill
%%%%\end{itemize}
%%%%\end{frame}
%%\begin{frame}{\heteroprioD: \heteroprio for Task Graphs}
%%
%%  \begin{itemize}
%%  \item Ready tasks are ranked by acceleration factor: $\rho_T$ = $\text{CPU time(T)/ GPU time(T)}$
%%  \item \emph{Favorable} tasks: those which run faster on $W$ than on another resource
%%  \end{itemize}
%%  
%%  \begin{block}{Presentation of the algorithm}
%%    \algrenewcommand\algorithmicwhile{\textbf{When}}
%%    	\begin{algorithmic}
%%          \While{a worker $W$ is idle}
%%          \State Pick the most favorable ready task (highest or lowest acceleration factor)
%%          \If{no favorable task is ready}
%%          \State Pick\textcolor{red}{*} a currently running task which would be finished earlier if started now on $W$
%%          \EndIf
%%          \State Schedule the picked task on $W$
%%          \EndWhile
%%        \end{algorithmic}
%%
%%  \end{block}
%%
%%  \medskip
%%  
%%  \raggedright
%%  \footnotesize
%%\textcolor{red}{*} Theoretical results do not depend on the pick
%%ordering, in practice it is better to pick the highest priority task
%%  
%%\end{frame}
%%
%%
%%\begin{frame}{\heteroprioD: Running Example}
%%\begin{figure}
%%	\vspace*{-0.5cm}
%%	\begin{columns}
%%		\begin{column}{0.275\linewidth}
%%	\begin{block}{\footnotesize Precedence Constraints}
%%		\begin{center}
%%	\begin{tikzpicture}[scale=0.60, every node/.style={transform shape}]
%%%%\draw[fill=cyan] (0,0) circle (0.8cm);
%%\tikzstyle{task}=[circle, draw, minimum size=10mm]
%%\node (d) at (0,0)[task, fill=babyblueeyes] {$D$};
%%\node (e) at (2.5,0)[task, fill=pastelviolet] {$E$};
%%\draw[thick, ->] (d.east) -- (e);
%%
%%\node (a) at (0,2)[task, fill=pastelyellow] {$A$};
%%\node (b) at (2.5,2.75)[task, fill=pastelgreen] {$B$};
%%\node (c) at (2.5,1.25)[task, fill=pastelred] {$C$};
%%
%%\draw[thick, ->] (a.east) -- (b);
%%\draw[thick, ->] (a.east) -- (c);
%%
%%\path (-0.1, -0.8) -- (2.5, -0.8); 
%%\end{tikzpicture}
%%		\end{center}
%%	\end{block}
%%		\end{column}
%%	\begin{column}{0.7\linewidth}
%%\begin{block}{}{\footnotesize
%%\begin{itemize}
%%%%	\item $p_T$ (resp. $q_T$): processing times of task $T$ on CPU (resp. GPU)
%%%%	\item $\frac{p_A}{q_A} > \frac{p_B}{q_B} > \frac{p_C}{q_C} > 1 > \frac{p_E}{q_E}$
%%	\item $\rho_A > \rho_B > \rho_C > \rho_D > 1 > \rho_E$
%%	\item Processing of $B$ and $C$ first start on GPU and CPU
%%	  respectively
%%        \item When $D$ ends:
%%	  \begin{itemize}
%%          \item $C$ is spoliated by a GPU ($E$ is ready, but not favorable)
%%	  \item $E$ is scheduled on the CPU available after the spoliation
%%          \end{itemize}
%%\end{itemize}
%%}\end{block}
%%	\end{column}
%%	\end{columns}
%%\subfloat[\heteroprioD schedule when task $C$ is first scheduled.]{
%%	\label{fig:hetprio.example.first}
%%	\begin{tikzpicture}[scale=0.60, every node/.style={transform shape}]
%%	\tikzstyle{task}=[draw, minimum height=0.5cm,
%%	inner sep=0pt, anchor=south west]
%%	\tikzstyle{legend}=[below=0.4, anchor=mid] 
%%	%%	\draw[color=red] (2.8, -0.3) -- +(0, 3.25);
%%	
%%	\foreach \y/\l in { 0/{0.4/0, 0.3/0.4, 0.7/0.7}, 
%%		0.5/{0.3/0, 0.4/0.3, 0.3/0.7},
%%		1.25/{0.5/0, 0.3/0.5, 1.2/0.8, 1.5/2.0},
%%		1.75/{0.8/0, 0.4/0.8, 0.7/1.2, 0.5/1.9},
%%		2.25/{0.6/0, 0.4/0.6, 1.1/1.0, 1.0/2.1},
%%	} {
%%		\foreach \w/\x in \l
%%		\node at (\x, \y)[task, minimum width=\w cm, fill=white] {};
%%	}
%%	
%%	\node at (1.0, 0.5)[task, minimum width=1.2cm, fill=pastelyellow]{$A$};
%%	\node at (2.2, 0.5)[task, minimum width=1.0cm, fill=pastelgreen]{$B$};
%%	\node at (1.4, 0)[task, minimum width=1.4cm, fill=babyblueeyes]{$D$};
%%	\node at (2.4, 1.75)[task, minimum width=1.6cm, fill=pastelred]{$C$};
%%	
%%	\draw[->] (-0.3, -0.3) -- (0, -0.3) -- node[pos=0, legend] {$0$}
%%	node[pos=0.57, legend] {current time} (4.2, -0.3) node[right] {$t$}; 
%%	\draw[dashed] (2.4,-0.3) -- +(0, 3.25);
%%	\foreach \x in {0, 2.4} \draw (\x, -0.3) -- (\x, -0.4);
%%	\node at (-0.3, 2.0) [anchor=south, rotate=90] {CPUs}; 
%%	\node at (-0.3, 0.5) [anchor=south, rotate=90] {GPUs}; 
%%	
%%	
%%	\end{tikzpicture}
%%} 
%%\hfill
%%\subfloat[\heteroprioD schedule after spoliation of task $C$.]{
%%	\label{fig:hetprio.example.spoliation}
%%	\begin{tikzpicture}[scale=0.60, every node/.style={transform shape}]
%%	\tikzstyle{task}=[draw, minimum height=0.5cm,
%%	inner sep=0pt, anchor=south west]
%%	\tikzstyle{legend}=[below=0.4, anchor=mid] 
%%	%%	\draw[color=red] (2.8, -0.3) -- +(0, 3.25);
%%	\foreach \y/\l in { 0/{0.4/0, 0.3/0.4, 0.7/0.7}, 
%%		0.5/{0.3/0, 0.4/0.3, 0.3/0.7},
%%		1.25/{0.5/0, 0.3/0.5, 1.2/0.8, 1.5/2.0},
%%		1.75/{0.8/0, 0.4/0.8, 0.7/1.2, 0.5/1.9},
%%		2.25/{0.6/0, 0.4/0.6, 1.1/1.0, 1.0/2.1},
%%	} {
%%		\foreach \w/\x in \l
%%		\node at (\x, \y)[task, minimum width=\w cm] {};
%%	}
%%	\draw[dashed] (4.0,-0.3) -- +(0, 3.25);
%%	\node at (1.0, 0.5)[task, minimum width=1.2cm, fill=pastelyellow]{$A$};
%%	\node at (2.2, 0.5)[task, minimum width=1.0cm, fill=pastelgreen]{$B$};
%%	\node at (1.4, 0)[task, minimum width=1.4cm, fill=babyblueeyes]{$D$};
%%	\node at (2.8, 0)[task, minimum width=0.6cm, fill=pastelred]{$C$};
%%	%%	\node at (2.4, 1.75)[task, minimum width=1.4cm, , fill=blue!10, pattern=north east lines]{aborted};
%%	
%%	\draw[->] (-0.3, -0.3) -- (0, -0.3) -- node[pos=0, legend] {$0$}
%%	node[pos=0.66, legend] {current time}  (4.2, -0.3) node[right] {$t$}; 
%%	\draw[dashed] (2.8,-0.3) -- +(0, 3.25);
%%	\foreach \x in {0, 2.8, 4.0} \draw (\x, -0.3) -- (\x, -0.4);
%%	
%%	\node at (2.4, 1.75)[task, minimum width=1.6cm,  pattern=north
%%	east lines, pattern color=patternblue](aborted){};
%%	\node at (aborted.center) [fill=white, inner sep=2pt]{aborted};
%%	
%%	%%	\node at (-0.3, 2.0) [anchor=south, rotate=90] {CPUs}; 
%%	%%	\node at (-0.3, 0.5) [anchor=south, rotate=90] {GPUs}; 
%%	
%%	\end{tikzpicture}
%%}
%%\hfill
%%\subfloat[\heteroprioD schedule with first response of task $E$.]{
%%	\label{fig:hetprio.example.last}
%%	\begin{tikzpicture}[scale=0.60, every node/.style={transform shape}]
%%	\tikzstyle{task}=[draw, minimum height=0.5cm,
%%	inner sep=0pt, anchor=south west]
%%	\tikzstyle{legend}=[below=0.4, anchor=mid] 
%%	%%	\draw[color=red] (2.8, -0.3) -- +(0, 3.25);
%%	\foreach \y/\l in { 0/{0.4/0, 0.3/0.4, 0.7/0.7}, 
%%		0.5/{0.3/0, 0.4/0.3, 0.3/0.7},
%%		1.25/{0.5/0, 0.3/0.5, 1.2/0.8, 1.5/2.0},
%%		1.75/{0.8/0, 0.4/0.8, 0.7/1.2, 0.5/1.9},
%%		2.25/{0.6/0, 0.4/0.6, 1.1/1.0, 1.0/2.1},
%%	} {
%%		\foreach \w/\x in \l
%%		\node at (\x, \y)[task, minimum width=\w cm] {};
%%	}
%%	
%%	%%	\draw[dashed] (4.0,-0.3) -- +(0, 3.25);
%%	\node at (1.0, 0.5)[task, minimum width=1.2cm, fill=pastelyellow]{$A$};
%%	\node at (2.2, 0.5)[task, minimum width=1.0cm, fill=pastelgreen]{$B$};
%%	\node at (1.4, 0)[task, minimum width=1.4cm, fill=babyblueeyes]{$D$};
%%	\node at (2.8, 0)[task, minimum width=0.6cm, fill=pastelred]{$C$};
%%	
%%	%%	\node at (2.4, 1.75)[task, minimum width=1.4cm, , fill=blue!10, pattern=north east lines]{aborted};
%%	\node at (2.4, 1.75)[task, minimum width=0.4cm,  pattern=north east lines, pattern color=patternblue]{};
%%	\node at (2.8, 1.75)[task, minimum width=0.8cm,  fill=pastelviolet]{$E$};
%%	
%%	\draw[->] (-0.3, -0.3) -- (0, -0.3) -- node[pos=0, legend] {$0$}
%%	node[pos=0.66, legend] {current time}   (4.2, -0.3) node[right] {$t$}; 
%%	\foreach \x in {0, 2.8} \draw (\x, -0.3) -- (\x, -0.4);
%%	\draw[dashed] (2.8,-0.3) -- +(0, 3.25);
%%	%%	\node at (-0.3, 2.0) [anchor=south, rotate=90] {CPUs}; 
%%	%%	\node at (-0.3, 0.5) [anchor=south, rotate=90] {GPUs}; 	
%%	\end{tikzpicture}
%%} 
%%\end{figure}
%%\end{frame}
%%\setcounter{subfigure}{0}
%%\begin{frame}{\heteroprioD Schedule}
%%%%\begin{itemize}
%%%%	\item 
%%%%\end{itemize}
%%\begin{columns}
%%\begin{column}{0.45\linewidth}
%%\begin{block}{\footnotesize $T$ completes on its unfavorable resource}
%%	\begin{center}
%%		\begin{tikzpicture}[scale=0.75, every node/.style={transform shape}]
%%%%\draw[snake=brace] (a0.west |- a0.north) -- (t5.north -| t5.east) node[rectangle, auto=left,midway,anchor=south, inner sep=4pt] {$1+\frac{m}{n}$ repetitions};
%%
%%\tikzstyle{task}=[draw, minimum height=0.5cm,
%%inner sep=0pt, anchor=south west]
%%\tikzstyle{legend}=[below=0.4, anchor=mid] 
%%%%	\draw[color=red] (2.8, -0.3) -- +(0, 3.25);
%%\foreach \y/\l in { 0/{1.5/0, 2.4/1.5, 0.5/3.9}, 
%%	0.5/{1.0/0, 1.0/1.0, 1.8/2.0, 0.3/3.8},
%%	1.25/{1.5/0, 1.7/1.5, 1.0/3.2},
%%	1.75/{1.0/0, 3.1/1.0},
%%	2.25/{1.5/0, 2.5/1.5, 0.5/4.0},
%%} {
%%	\foreach \w/\x in \l
%%	\node at (\x, \y)[task, minimum width=\w cm] {};
%%}
%%
%%%%	\draw[dashed] (4.0,-0.3) -- +(0, 3.25);
%%\node at (1.5, 2.25)[task, minimum width=2.5cm, fill=pastelviolet]{$T$};
%%\node at (0, 1.25)[task, minimum width=1.5cm,  pattern=north
%%east lines, pattern color=patternblue]{};
%%\node at (1, 1.75)[task, minimum width=1cm,  pattern=north east
%%lines, pattern color=patternblue]{};
%%
%%\node at (5.5, 2.25)[task, minimum width=2.5cm, fill=pastelviolet]{$T$ on CPU};
%%\node at (5.5, 0.5)[task, minimum width=.5cm, fill=pastelviolet, name=TGPU]{};
%%\node at (TGPU)[right=0.2cm]{$T$ on GPU}; 
%%
%%%%\draw[->] (-0.3, -0.3) -- (0, -0.3) -- node[pos=0, legend] {$0$} (4.2, -0.3) node[right] {$t$}; 
%%%%\foreach \x in {0} \draw (\x, -0.3) -- (\x, -0.4);
%%%%\draw[|-|] (0,0) -- (3,0) node[rectangle, auto=left,midway,anchor=north, inner sep=4pt] {$1+\frac{m}{n}$ repetitions};
%%\end{tikzpicture}
%%	\end{center}
%%\end{block}
%%\end{column}
%%\begin{column}{0.45\linewidth}
%%\begin{block}{\footnotesize $T$ is replaced with $T'$ and $T''$}
%%	\begin{center}
%%		\begin{tikzpicture}[scale=0.75, every node/.style={transform shape}]
%%%%\draw[snake=brace] (a0.west |- a0.north) -- (t5.north -| t5.east) node[rectangle, auto=left,midway,anchor=south, inner sep=4pt] {$1+\frac{m}{n}$ repetitions};
%%
%%\tikzstyle{task}=[draw, minimum height=0.5cm,
%%inner sep=0pt, anchor=south west]
%%\tikzstyle{legend}=[below=0.4, anchor=mid] 
%%%%	\draw[color=red] (2.8, -0.3) -- +(0, 3.25);
%%\foreach \y/\l in { 0/{1.5/0, 2.4/1.5, 0.5/3.9}, 
%%	0.5/{1.0/0, 1.0/1.0, 1.8/2.0, 0.3/3.8},
%%	1.25/{1.5/0, 1.7/1.5, 1.0/3.2},
%%	1.75/{1.0/0, 3.1/1.0},
%%	2.25/{1.5/0, 2.5/1.5, 0.5/4.0},
%%} {
%%	\foreach \w/\x in \l
%%	\node at (\x, \y)[task, minimum width=\w cm] {};
%%}
%%
%%%%	\draw[dashed] (4.0,-0.3) -- +(0, 3.25);
%%\node at (1.5, 2.25)[task, minimum width=2.0cm, fill=pastelred]{$T'$};
%%\node at (3.5, 2.25)[task, minimum width=0.5cm, fill=pastelgreen]{$T''$};
%%\node at (0, 1.25)[task, minimum width=1.5cm,  pattern=north east lines, pattern color=patternblue]{};
%%\node at (1, 1.75)[task, minimum width=1cm,  pattern=north east lines, pattern color=patternblue]{};
%%
%%\node at (5.5, 2.25)[task, minimum width=2cm, fill=pastelred]{$T'$ on CPU};
%%\node at (5.5, 0.5)[task, minimum width=.01cm, color=pastelred, name=TpGPU]{};
%%\node at (TpGPU)[right=0.2cm]{$T'$ on GPU}; 
%%
%%\node at (5.5, 1.5)[task, minimum width=0.5cm, fill=pastelgreen, name=TppCPU]{};
%%\node at (TppCPU)[right=0.2cm]{$T''$ on CPU};
%%\node at (5.5, 0)[task, minimum width=.5cm, fill=pastelgreen, name=TppGPU]{};
%%\node at (TppGPU)[right=0.2cm]{$T''$ on GPU}; 
%%
%%                \end{tikzpicture}
%%	\end{center}
%%\end{block}
%%\end{column}
%%\end{columns}
%%\begin{flushleft}{\footnotesize
%%We divide the modified schedule into different types of segments and try to bound each segment separately.
%%\begin{itemize}
%%	\item[$\bullet$] $ST_A$: Segments where at least one worker is idle on
%%	both types of resources
%%	\item[$\bullet$] $ST_B$: Segments where at least one type of resource
%%	is fully busy and at least one task runs on its unfavorable resource
%%	(either a spoliated or a task of type $T'$)
%%	\item[$\bullet$] $ST_C$: Segments where at least one type of resource
%%	is fully busy, and all tasks are on their favorable resources
%%\end{itemize} 
%%}\end{flushleft}
%%\end{frame}
%%\begin{frame}{Different Segments of a Modified \heteroprioD Schedule}
%%	\begin{center}
%%	\begin{tikzpicture}[scale=0.615, every node/.style={transform shape}]
%%	\tikzstyle{task}=[draw, minimum height=0.5cm,
%%	inner sep=0pt, anchor=south west]
%%	\tikzstyle{legend}=[below=0.4, anchor=mid] 
%%	%%	\draw[color=red] (2.8, -0.3) -- +(0, 3.25);
%%	\foreach \y/\l in { 0/{1.0/0, 1.8/1.0, 1.8/2.8, 0.65/4.6, 2.1/9.3, 0.6/11.4, 1.8/13, 0.4/14.8, 
%%			1.6/15.2, 0.5/16.8, 1.6/17.3}, 
%%		0.5/{0.5/0, 2.0/0.5, 1.0/2.5, 0.8/3.5, 0.95/4.3, 1.05/5.25, 1.9/6.3, 1.3/8.2, 1.0/9.5, 2.1/10.5, 
%%			1.9/12.6, 1.5/14.5, 1.0/16},
%%		1.25/{1.0/0, 2.0/1.0, 1.25/4.5, 1.0/9.3, 2.2/10.3, 1.3/12.5, 1.0/13.8, 1.4/14.8, 0.8/16.2},
%%		1.75/{5.25/0, 2.1/5.9, 1.3/8.0, 2.0/9.3, 1.7/11.3, 2.0/13.0, 1.5/15, 1.2/16.5},
%%		2.25/{2.0/0, 2.5/2.0, 1.4/4.5, 3.1/5.9, 1.2/9, 1.8/10.2, 1.8/12, 3.2/13.8, 0.4/17.0, 0.6/17.4},
%%	} {
%%		\foreach \w/\x in \l
%%		\node at (\x, \y)[task, minimum width=\w cm] {};
%%	}
%%	
%%	\node at (0, 1.75)[task, minimum width=5.25cm, pattern=north east lines, pattern 
%%	color=patternblue]{$T_1$};
%%	\node at (5.25, 0.5)[task, minimum width=1.05cm]{$T_1$};
%%	
%%	\node at (13.8, 1.25)[task, minimum width=1cm, pattern=north east lines, pattern 
%%	color=patternblue]{$T_3$};
%%	\node at (14.8, 0)[task, minimum width=0.4cm]{$T_3$};
%%	\node at (13.8, 2.25)[task, minimum width=3.2cm, fill=pastelred]{$T_2'$};
%%	\node at (17, 2.25)[task, minimum width=0.4cm, fill=pastelgreen]{$T_2''$};
%%	
%%	\draw[dotted] (0.0, -0.75) -- +(0, 3.85);
%%	\draw[dotted] (5.25, -0.75) -- +(0, 3.85);
%%	\draw[dotted] (9.3, -0.75) -- +(0, 3.85);
%%	\draw[dotted] (13.8, -0.75) -- +(0, 3.85);
%%	\draw[dotted] (17.0, -0.75) -- +(0, 3.85);
%%	\draw[dotted] (18.9, -0.75) -- +(0, 3.85);
%%	
%%	\draw[|-|, thin] (0, -0.5) -- (5.25, -0.5) node[rectangle,
%%	auto=left,midway,anchor=north, inner sep=4pt] {segment of $ST_B$};
%%	\draw[|-|, thin] (5.25, -0.5) -- (9.3, -0.5) node[rectangle,
%%	auto=left,midway,anchor=north, inner sep=4pt] {segment of $ST_A$};
%%	\draw[|-|, thin] (9.3, -0.5) -- (13.8, -0.5) node[rectangle,
%%	auto=left,midway,anchor=north, inner sep=4pt] {segment of $ST_C$};
%%	\draw[|-|, thin] (13.8, -0.5) -- (17.0, -0.5) node[rectangle,
%%	auto=left,midway,anchor=north, inner sep=4pt] {segment of $ST_B$};
%%	\draw[|-|, thin] (17.0, -0.5) -- (18.9, -0.5) node[rectangle,
%%	auto=left,midway,anchor=north, inner sep=4pt] {segment of $ST_A$};
%%	
%%	
%%	\node at (-0.3, 2.0) [anchor=south, rotate=90] {CPUs}; 
%%	\node at (-0.3, 0.5) [anchor=south, rotate=90] {GPUs}; 
%%	
%%	\end{tikzpicture}
%%	{\small
%%	\begin{itemize}
%%		\item Platform is composed of $m$ CPUs and $n$ GPUs
%%		\item $\HE$ and $\copt$ represent \heteroprioD and \textit{Optimal} makespans respectively
%%		\item We prove that $\HE = |ST_A| + |ST_B| + |ST_C| \le \left(2+ \max(\frac{m}{n}, \frac{n}{m})\right)\copt$
%%	\end{itemize}
%%}
%%\end{center}
%%\end{frame}
%%
%%%%\begin{frame}{\heteroprioD Worst Case Example}
%%%%\begin{block}{Task Graph}
%%%%	\begin{center}
%%%%		\begin{tikzpicture}[circle, every
%%%%		node/.style={font=\footnotesize}, inner sep=1pt]
%%%%		\foreach \i in {0, 2, 5} {
%%%%			\begin{scope}[shift={(\i, 0)}]
%%%%			\node (a\i) at (0, 3) {$A$};
%%%%			\node (t\i) at (1, 3) {$T_1$};
%%%%			\node (u\i) at (1, 2.5) {$T_2$};
%%%%			\node (btop\i) at (2, 2.5) {$B$};
%%%%			\node (bdown\i) at (2, 1.5) {$B$};
%%%%			\end{scope}
%%%%			
%%%%			\draw[very thick, loosely dotted] (btop\i) -- (bdown\i);
%%%%			\draw[->] (a\i) edge (t\i) edge (u\i)
%%%%			(u\i) edge (btop\i) edge (bdown\i);
%%%%			\draw[snake=brace] (btop\i.north -| btop\i.east) --
%%%%			(bdown\i.south -| bdown\i.east) node[midway, auto=left, inner
%%%%			sep=4pt] {$n$ tasks};
%%%%		}
%%%%		\node (a7) at (7, 3) {$A$}; 
%%%%		\draw[->] (t0) edge (a2) (t5) edge (a7);
%%%%		\draw[very thick, loosely dotted] (t2) -- (a5); 
%%%%		
%%%%		\draw[snake=brace] (a0.west |- a0.north) -- (t5.north -|
%%%%		t5.east) node[rectangle, auto=left,midway,anchor=south, inner sep=4pt] {$1+\frac{m}{n}$ repetitions};
%%%%		\end{tikzpicture}
%%%%	\end{center}
%%%%\end{block}
%%%%
%%%%\begin{block}{Task Description}
%%%%\begin{center}
%%%%	\begin{tabular}{|c|c|c|c|}
%%%%		\hline
%%%%		Task Name & \# & GPU time &CPU time \\ 
%%%%		$A$ & $\frac{m}{n} + 2$ & $2\epsilon$ & $1-3\epsilon$\\
%%%%		$T_1$ & $\frac{m}{n} + 1$ & $5\epsilon$ & $4\epsilon$\\
%%%%		$T_2$ & $\frac{m}{n} + 1$ & $2\epsilon$ & $3\epsilon$\\
%%%%		$B$ & $m+n$ & $1-\epsilon$ & $1$ \\ 
%%%%		\hline
%%%%	\end{tabular}
%%%%\end{center}
%%%%\end{block}
%%%%\end{frame}
%%
%%\tikzset{task/.style args={#1start#2length#3res#4}{draw, inner
%%		sep=0pt, fill=gray!25!white, label=center:#1, fit={(#2*0.2,#4*0.35)
%%			(#2*0.2+#3*0.2,#4*0.35+0.35)}}}
%%		
%%\newcommand{\taskA}[1]{\node[task=$A$ start #1 length 7 res 0]{};}
%%\newcommand{\taskBs}[1]{\node[task=$B$ start #1 length 9 res
%%	-1,name=top]{};
%%	\node[task=$B$ start #1 length 9 res -4
%%	, name=bot]{}; \draw[loosely dotted, very thick]
%%	(top) -- (bot);
%%}
%%\newcommand{\taskT}[1]{\node[task=$T_1$ start #1 length 4 res 0]{};}
%%\newcommand{\taskU}[1]{\node[task=$T_2$ start #1 length 2 res -1]{};}
%%\newcommand{\taskAA}[1]{\node[task=$A$ start #1 length 2 res -1]{};}
%%\newcommand{\taskBBs}[1]{\node[task=$B$ start #1 length 9 res
%%	-1,name=top]{};
%%	\node[task=$B$ start #1 length 9 res -4
%%	, name=bot]{}; \draw[loosely dotted, very thick]
%%	(top) -- (bot);
%%	\node[task=$B$ start #1 length 9 res 0,name=botC]{};
%%	\node[task=$B$ start #1 length 9 res 4,name=topC]{};
%%	\draw[loosely dotted, very thick] (botC) -- (topC);
%%}
%%
%%\begin{frame}{\heteroprioD Worst Case Example}
%%\vspace*{-0.25cm}
%%\begin{columns}
%%	\begin{column}{0.475\linewidth}
%%\begin{block}{\footnotesize Task Graph}
%%	\begin{center}
%%		\begin{tikzpicture}[scale=0.625, circle, every
%%		node/.style={font=\tiny}, inner sep=1pt]
%%		\foreach \i in {0, 2, 5} {
%%			\begin{scope}[shift={(\i, 0)}]
%%			\node (a\i) at (0, 3) {$A$};
%%			\node (t\i) at (1, 3) {$T_1$};
%%			\node (u\i) at (1, 2.5) {$T_2$};
%%			\node (btop\i) at (2, 2.5) {$B$};
%%			\node (bdown\i) at (2, 1.5) {$B$};
%%			\end{scope}
%%			
%%			\draw[very thick, loosely dotted] (btop\i) -- (bdown\i);
%%			\draw[->] (a\i) edge (t\i) edge (u\i)
%%			(u\i) edge (btop\i) edge (bdown\i);
%%			\draw[snake=brace] (btop\i.north -| btop\i.east) --
%%			(bdown\i.south -| bdown\i.east) node[midway, auto=left, inner
%%			sep=4pt] {$n$ tasks};
%%		}
%%		\node (a7) at (7, 3) {$A$}; 
%%		\draw[->] (t0) edge (a2) (t5) edge (a7);
%%		\draw[very thick, loosely dotted] (t2) -- (a5); 
%%		
%%		\draw[snake=brace] (a0.west |- a0.north) -- (t5.north -|
%%		t5.east) node[rectangle, auto=left,midway,anchor=south, inner sep=4pt] {$1+\frac{m}{n}$ repetitions};
%%		\end{tikzpicture}
%%	\end{center}
%%\end{block}
%%\end{column}
%%\begin{column}{0.475\linewidth}
%%\begin{block}{\footnotesize Task Description}{\scriptsize
%%	\begin{center}
%%		\begin{tabular}{|c|c|c|c|}
%%			\hline
%%			Task & \# & GPU time &CPU time \\ 
%%			$A$ & $\frac{m}{n} + 2$ & $2\epsilon$ & $1-3\epsilon$\\
%%			$T_1$ & $\frac{m}{n} + 1$ & $5\epsilon$ & $4\epsilon$\\
%%			$T_2$ & $\frac{m}{n} + 1$ & $2\epsilon$ & $3\epsilon$\\
%%			$B$ & $m+n$ & $1-\epsilon$ & $1$ \\ 
%%			\hline
%%		\end{tabular}
%%	\end{center}}
%%\end{block}
%%	\end{column}
%%\end{columns}
%%%%\end{frame}
%%%%
%%%%\begin{frame}{\heteroprioD Worst Case Example}
%%	\begin{columns}
%%		\begin{column}{0.475\linewidth}
%%\begin{block}{\footnotesize A Possible \heteroprioD Schedule}
%%	    \begin{center}
%%		\begin{tikzpicture}[scale=0.625, every node/.style={font=\tiny},
%%		inner sep=1pt]
%%		\draw[->] (-0.2, -1.4) -- (8.5, -1.4) node[below] {$t$};
%%		\draw (0, -1.4) -- (0, 1.8);
%%		\node at (-0.3, -0.7)[rotate=90] {$n$ GPUs};
%%		\node at (-0.3, 0.9)[rotate=90] {$m$ CPUs};
%%		\draw[dashed,gray,thick] (-0.2, 0) -- (8.5, 0);
%%		
%%		\node[task=$A$ start 0 length 2 res -1]{};
%%		\taskT{2}
%%		\taskU{2}
%%		\taskBs{4}
%%		\taskA{6}
%%		\taskT{13}
%%		\taskU{13}
%%		\taskBs{15}
%%		\taskA{17}
%%		
%%		\taskT{30}
%%		\taskU{30}
%%		\taskBs{32}
%%		\taskA{34}
%%		\foreach \y in {0.15, -0.15, -1.25} 
%%		\draw[loosely dotted, very thick] (4.9, \y) -- (5.9, \y);
%%		\end{tikzpicture}
%%	\end{center}
%%\end{block}
%%\centering $\HE = (\frac{m}{n}+1)(1+\epsilon) +2\epsilon$
%%              \end{column}
%%\begin{column}{0.475\linewidth}
%%\begin{block}{\footnotesize A Better Schedule}
%%	    \begin{center}		
%%		\begin{tikzpicture}[scale=0.625, every node/.style={font=\tiny},
%%		inner sep=1pt]
%%		\draw[->] (-0.2, -1.4) -- (8.5, -1.4) node[below] {$t$};
%%		\draw (0, -1.4) -- (0, 1.8);
%%		\node at (-0.3, -0.7)[rotate=90] {$n$ GPUs};
%%		\node at (-0.3, 0.9)[rotate=90] {$m$ CPUs};
%%		\draw[dashed,gray,thick] (-0.2, 0) -- (8.5, 0);
%%		
%%		\taskAA{0}
%%		\taskT{2}
%%		\taskU{2}
%%		\taskAA{6}
%%		\taskT{8}
%%		\taskU{8}
%%		
%%		\foreach \y in {0.15, -0.15} 
%%		\draw[loosely dotted, very thick] (2.5, \y) -- (3.9, \y);
%%		
%%		\taskAA{20}
%%		\taskT{22}
%%		\taskU{22}
%%		\taskAA{26}
%%		\taskBBs{28}
%%		\end{tikzpicture}
%%	\end{center}
%%\end{block}
%%\centering $\copt \le 2\epsilon + 6\epsilon \left(\frac{m}{n} +1\right) +
%%1$
%%\end{column}
%%\end{columns}
%%\vspace*{-0.15cm}
%%\begin{center}
%%$\lim_{\epsilon \to 0} \frac{\HE}{\copt} \ge \frac{m}{n} + 1$
%%\end{center}
%%
%%\end{frame}
%%
%%\begin{frame}{State-of-the-art Approaches \& Experimental Setup}
%%\vspace*{-0.15cm}
%%	\begin{center}
%%{\scriptsize
%%\begin{tabular}{|l|l|c|p{5.835cm}|}
%%	\hline
%%	Algorithm & Type & Approx. ratio & Description\\ \hline
%%	HLP & Offline & $6$ & Solve linear program to find task allocations and schedule tasks using list scheduling algorithm\\ \hline
%%	ER-LS & Online & $4\sqrt{max(\frac{m}{n}, \frac{n}{m})}$ & Comparison of CPU time of a task and its earliest completion time on GPU, and average processing time of a task on CPU and GPU\\ \hline
%%	QA & Online & $1 +2\sqrt{max(\frac{m}{n}, \frac{n}{m})}$ & Modified version of ER-LS, only the second decision rule of ER-LS\\ \hline
%%	HEFT & Offline & $\Omega(max(\frac{m}{n}, \frac{n}{m}))$ & The highest priority task is scheduled on the resource which finishes it at the earliest time\\ \hline
%%%%	QA & Online & $1 +2\sqrt{max(\frac{m}{n}, \frac{n}{m})}$ & M\\ \hine
%%%%	HEFT & Offline & $\Omega(max(\frac{m}{n}, \frac{n}{m})$ & The highest priority task is scheduled on the resource which finishes it at the earliest time\\ \hine
%%	ECT & Online & $\Omega(max(\frac{m}{n}, \frac{n}{m}))$ & Online version of HEFT\\ \hline
%%\end{tabular}}
%%	\end{center}
%%\vspace*{-0.3cm}
%%{\footnotesize\begin{block}{Experimental Setup}
%%		\vspace*{-0.1cm}
%%	\begin{itemize}
%%	\setlength\itemsep{0.125em}
%%	\item Task based Cholesky, QR and LU  kernels of Chameleon library, with TileSize = 960
%%	%%		\item StarPU runtime system for actual execution
%%	\item \emph{sirocco} platform description (single node)
%%	\begin{itemize}
%%		\item 24 CPU cores (\textbf{20 computational cores}) of Intel® Xeon® E5-2680 
%%		\item \textbf{4} Nvidia K40-M \textbf{GPUs}   
%%	\end{itemize}
%%	\item Task timings are obtained by executing kernels on each type of computing resource of \emph{sirocco}
%%\end{itemize}
%%\end{block}
%%}\end{frame}
%%
%%\begin{frame}{Schedulers Performance for Task Graphs}
%%\begin{columns}
%%	\begin{column}{0.525\linewidth}
%%%%		\todo[inline]{SK2LED: remove results corresponding to 20 and 60 number of CPUs}
%%		\begin{block}{}
%%		\begin{center}
%%%%		\includegraphics[scale=0.4]{./bordeaux-Ratio.pdf}
%%		\includegraphics[scale=0.4]{./transparent-bordeaux-Ratio.pdf}
%%		\end{center}
%%		\end{block}	
%%	\end{column}
%%\begin{column}{0.465\linewidth}{\scriptsize
%%		\begin{center}
%%\begin{itemize}
%%	\item Normalized makespan is the ratio of the
%%	makespan obtained by each algorithm to the lower bound
%%	\item  Number of 960 $\times$ 960 tiles vary from 4 to 60
%%	\item Each row shows simulated results for a platform with 4 GPUs and a number of CPUs (4, 8, 12, 20, 40, 60)
%%	\item \heteroprioD outperforms other algorithms in most cases
%%	\item HLP performance is close to \heteroprioD for large task graphs
%%	\item HEFT performance is close to \heteroprioD for small task graphs
%%\end{itemize}
%%		\end{center}
%%}\end{column}
%%\end{columns}
%%\end{frame}

 \begin{frame}{Conclusion and Future Work}
 \begin{block}{Conclusion}
\begin{itemize}
	\item Proposed \heteroprioD algorithm, an extension of \heteroprio, for task graphs
	\item Presented theoretical approximation guarantees and worst case examples for \heteroprioD
	\item Empirical evaluation of \heteroprioD with state-of-the-art schedulers
\end{itemize}
 \end{block}
\vfill
\begin{block}{Future Work}
\begin{itemize}
	\item Model locality and data transfer costs in \heteroprioD
	\item Evaluate \heteroprioD for machine learning applications
\end{itemize}
 \end{block}
\end{frame}


%%\begin{frame}
%%\huge{\centerline{Thank You!}}
%%\end{frame}

\end{document}
