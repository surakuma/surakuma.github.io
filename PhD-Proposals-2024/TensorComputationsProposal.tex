\documentclass[a4paper]{article}


\usepackage[margin=1in,footskip=0.25in]{geometry}
\usepackage{hyperref}
%\usepackage{url}
%opening
%%\title{Parallel Tensor Decompositions for Data Analytics}
%%\author{}
\date{}
\pagenumbering{gobble}
\begin{document}
%%\maketitle
\noindent\textbf{PhD subject}: \emph{Parallel Tensor Decompositions for Data Analytics}\newline
\textbf{Advisors}: Suraj Kumar and Bora Ucar\newline
\textbf{Location}: ROMA team, Inria Lyon, France
\section*{Context}
Tensors are multi dimensional arrays and used to store data in several domains, for example, data mining, neuroscience and computer vision. Tensor decompositions help to identify inherent structure of data, achieve data compression and enable various ways of data analysis. Working with tensors is challenging due to their large computational effort and memory requirements (amount of memory and computations grow exponentially in the number of dimensions). It is therefore necessary to work with patterns of tensor data. Using low dimensional structure of high dimensional data is a powerful approach in this context. Most tensor decompositions represent data in its low dimensional structures.

CP and Tucker are the widely used tensor decompositions for data analytics. Both decompositions can be viewed as high order generalization of singular value decomposition. The data generated by many experiments is so big that it is impossible to perform a CP/Tucker decomposition without parallel computations. Therefore, it is important to design parallel and scalable algorithms for both decompositions.



%%This PhD thesis will be held at Inria Lyon, located at LIP laboratory (ENS de Lyon), France in collaboration with :



\section*{Assignment}
Most existing tensor decomposition algorithms work with matrix (2-dimensional) representations of tensors at each step and rely on the parallelization of matrix operations. This approach neglects high dimensional properties of tensors and may not perform the entire computations efficiently~\cite{ABGKR-SIMAX-2024,BKR-IPDPS-2018}. The main objective of this PhD thesis is to design parallel approaches based on multidimensional properties of tensors for CP and Tucker decompositions. 


This PhD thesis will be held in the ROMA Inria team at LIP, ENS Lyon under the supervision of Suraj Kumar and Bora Ucar. We will also collaborate with Grey Ballard (Wake Forest University, USA).

\section*{Main Activities}

The candidate is expected to perform the following activities:
\begin{itemize}
	\item Analyze computation and communication costs of the existing approaches for CP and Tucker decompositions
	\item Study minimum communication and computation costs for both decompositions
	\item Design new algorithms based on multidimensional properties of tensors
	\item Implement the proposed algorithms for large scale high performance computing systems
\end{itemize}

\section*{Skills}

The candidate must have a Master's degree  in Computer Science, Computational Sciences, Applied Mathematics, or a related technical field.\newline
Familiarity with Linear Algebra computations and Parallel Programming will be much appreciated. 



%%\section*{Benefits Package}
%%\begin{itemize}
%%	\item Subsidized Lunches
%%	\item Partial reimbursement of public transport costs
%%	\item Social security coverage under conditions
%%\end{itemize}
%%
%%\section*{Remuneration}
%%	\textbf{1st and 2nd year}: 2 051 euros gross salary /month\newline
%%	\textbf{3rd year}: 2 158 euros gross salary / month

\bibliographystyle{IEEEtranS}
\footnotesize \bibliography{tensor-computations}


\end{document}
